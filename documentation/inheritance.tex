\documentclass[oops.tex]{subfiles}
\begin{document}
\section{Inheritance}

Its is capability of the class to \emph{derive properties and characteristics} 
from another(parent) class. \texttt{Sub Class / Derieved Class} is a class which 
inherits properties from Supper Class / Parent Class. \texttt{Supper Class /
Parent Class} is a class whose properties are begin inherited.\\

{\bf Why Inheritance ?}\\
It actually represents \texttt{real world phenomenon of inheritance in humans}.
Example you have some (but not all) characteristics(height, weight, eyes, ears, 
behavioural instinct) from your parents, and similar they have got those
characteristics from the grandparents and so on. Similarly in OOPS inheritance
helps in \texttt{deriving some (not necessarly all) characteristics} from the
parent / base class. The main advantage of this is \texttt{Code Reuseability}.

\lstinputlisting[% Without Inheritance 
style=codeStyleC++, caption = Without Inheritance Design]
{listings/inheritance/Noinheritance.cpp}

\begin{figure}[h]
    \begin{center}
        \caption{Without using Inheritance}
        \includegraphics[width=8cm]{without_inheritance.png}
    \end{center}
\end{figure}

\begin{figure}[h]
    \begin{center}
        \caption{With using Inheritance}
        \includegraphics[width=8cm]{with_inheritance.png}
    \end{center}
\end{figure}

\begin{center}
    \begin{tabular}{l | l}
        \texttt{Inheritance}            & \texttt{Relation}\\
        \hline
        \emph{Single Inheritance}       & deriving a class from single base class.\\
        \emph{Multiple Inheritance}     & deriving a class from multiple base class.\\
        \emph{Multilevel Inheritance}   & deriving a class from another derived class.\\
        \emph{Hierarchical Inheritance} & more than one derived class exits from a single base class.\\
        \emph{Hybrid Inheritance}       & combination of the above mentioned inheritance.\\
    \end{tabular}
\end{center}

{\bf Single Inheritance Example}\\
The code on the previous page is great example of single inheritance since
Iphone needs all basic functionality of phone plus provides some of its own
functionalities.

{\bf Multiple Inheritance Example}\\
Best example is of humans, any child inherits properties of it's mother and
father both which indicates multiple inheritance.

{\bf Multilevel Inheritance Example}\\
It's is extension for the single inheritance so, it would be like Iphone 7s
being inherited from Iphone 5s with additional functionality and so on.

{\bf Herarchical Inheritance Example}\\
Personal Example - In doing bittorrent project I had to write the Tracker
protocol, basically the resquest and response parameters for HTTP and UDP 
trackers are same, thus we can inherit from tracker base class the HTTP and UDP
tracker class just changing the underlying transport layer protocol for
communication that is application layer protocol remains the same.

{\bf Hybrid Inheritance Example}\\
It's bit uncommon to see, will discuss about it in diamond problem.


{\bf Objects and Inheritance}\\

\texttt{Order of Constructor and Destructor Call}\\
Consider the case of single inheritance as given below the call order of
constructor will the parent class be called first and then the constructor for 
derived class will be called. In case of destructor the derived class is called
first and then the destructor of the parent class is called.

\lstinputlisting[% Call Order
style=codeStyleC++, caption = Call Order of Constructor and Destructor]
{listings/inheritance/callOrder.cpp}

\begin{figure}[h]
    \begin{center}
        \caption{Call Oder in Inheritance}
        \includegraphics[width=9.5cm]{call_order.png}
    \end{center}
\end{figure}

\texttt{ What things extactly get inherited in derieved class ?}
\begin{itemize}
    \item Data members(public and protected) which are accessible from base class.
    \item Member functions(public and protected) which are accessible from base class.
    \item The base class’s constructors and destructor.
    \item The base class’s friends.\\
\end{itemize}

\texttt{ Object Slicing }\\

In C++ you can assign derived class object to base class object or you can even
reference derieved class using bases class pointers, and due to this there is 
slicing of the derivied class objects to form the base class. It is very simple
when you initialize base class object using derived class, slice off all
other data elements and copy only base class members in derived class.\\

\lstinputlisting[% Object Slicing
style=codeStyleC++, caption = Object Slicing]
{listings/inheritance/objectSlicing.cpp}

{\bf How can base class be initialized or reference to derieved class ?}\\ 
The simple answer is because its a \texttt{Base Class}, and C++ also
Base-to-Derieved class converstion and not the other way round. Basically when
you try to initialize a object by another object you try and make sure
whether its having memory allocated for object data members. It so happens
in the case of base class that it can be referenced or assigned by the derieved
class (derieved + base) and if you would think other way, you would see that
assignment of the derieved with base doesn't contain valid memory for the data
members.\\

\texttt{ Overriding Methods in Base Class }\\

\texttt{Function Overriding} is feature which allows use to delcare member
function in derieved class which is already present in the base class. This can
be very usefully in case if you can to change some functionalities of the base
class by creating newer version and overriding the older version of the
function.\\

\lstinputlisting[% Method Overriding
style=codeStyleC++, caption = Method Overriding]
{listings/inheritance/methodOverriding.cpp}

Function overriding is very different from function overloading,
since \texttt{function overriding occurs in case of inheritance} where derieved class 
is trying to create shadow over the base class functions. In case of function
overriding the \texttt{function signatures must exactly be same}. \texttt{During 
Class inheritance function overloading doesn't occur, function having same
name but different parameters will be forcefully overriden in base class.}
Even if both are complie time activities one must always keep in mind that

\begin{center}
    Function Overriding $!=$ Function Overloading\\
    There is no \texttt{overload resolution} between Base and Derived class.\\
    \texttt{Function Overloading} occurs withing a class member functions.\\
    \texttt{Function Overriding} occurs during inheritance of member functions.\\
\end{center}

Now let's see ambiguity case, like what happens when a object is derived from 
two base class both having the same function name. Which function will the 
derieved class call ?

\lstinputlisting[% Multiple Inheritance Ambiguity
style=codeStyleC++, caption = Multiple Inheritance Ambiguity]
{listings/inheritance/multipleInheritance.cpp}

When you complie the above code there is no complie time errors, however it
will result in runtime errors since there are two defination of function hello
available in the baz class. This situation is ambigous since there are more
than defination available and derived object is confused which function to 
link with. Inorder to resolve such ambiguities we make use of scope resolution
operators. Each base class function is wrapped in the namespace of itself and
since derieved class gets the all the functionality of public base class we 
can resolve the ambiguity using scope resolution operators.\\

\lstinputlisting[% Resolve Ambiguity
style=codeStyleC++, caption = Resolve Multiple Inheritance Ambiguity]
{listings/inheritance/resolve.cpp}

\newpage{}

{\bf Diamond Problem (Virtual Inheritance)}

Also know the Deadly Diamond of Death, its an problem leading to ambiguity when
class inheritance takes the following structure. The you can easily understand
ambiguity in the order in which constructor calls takes place if we instantiate
the object of class D.

\begin{figure}[h]
    \begin{center}
        \caption{Diamod Problem}
        \includegraphics[width=7cm]{diamond.png}
    \end{center}
\end{figure}


\lstinputlisting[% Diamond Problem 
style=codeStyleC++, caption = Resolve Multiple Inheritance Ambiguity]
{listings/inheritance/diamond.cpp}

Still after using the scope resolution operators its failing to bind to the
paritcular function, since its different kind of ambiguity as seen in previous 
case. Here in problem we have actaully two base class instance of parent in class
baz (base class instance from foo and from baz). \\

So inorder to resolve this paritcular ambiguity we make use of \texttt{Virtual
Inheritance} which helps in creating exactly one object instance of parent class 
every time the derived class object is instanciated.\\

\lstinputlisting[% Virtual Inheritance
style=codeStyleC++, caption = Virtual Inheritance]
{listings/inheritance/virtualInheritance.cpp}






\end{document}

