\documentclass[oops.tex]{subfiles}
\begin{document}
\section{Introduction}

{\bf Programming paradigms} - it is nothing but the \emph{style} of programming 
a certain problem. There are different categories in which programming languages
can be classified. There are four types of languages according to Programming
paradigms i.e. Imperative Languges (C), Object Oriented Programming Languges (C++,
Java, Python), Functional Programming Languges (Lisp) and Logical Programming
Languges (Prolog).\\

\emph{Object Oriented Programming} is type of programming paradigm in which as the name 
suggests objects are used for programming. The aim of object oriented programming 
(OOP) is to implement real life application entites though programming. The OOP binds 
the data and functions operations on the data together so that it forms a single
entity and no other function in the code can access that data.\\

{\bf Advantages of OOPS}\\
\emph{Represent Real World Objects/Entities}, \emph{Modular Code Structure},
\emph{Code Reuseability}\\

{\bf Disadvantages of OOPS}\\
\emph{Larger Program Size}, \emph{Slower than Procedural Languges}
\emph{Not suitable for all types of the problems}\\

\texttt{Class} : It is a user-defined prototype, which holds its own data members 
and member functions for which objects can be created. Data members are the
data variables and member functions are the methods used to manipulate these variables. 
Together these data members and member functions define the properties and behaviour 
of the objects in a class. Always remember class is like a \emph{blueprint} for
object.\\

\texttt{Objects} : Its nothing but an instance of the Class. When we delcare a
class no memory is allocated, however when we declare a object of the class or
in other words we create an instance of the class memory is allocated to that
object. An object is an entity which has some behaviours and charateristics.\\

\texttt{Encapsulation} : Encapsulation is defined as wrapping up of data and 
information under a single unit. Encapsulation also leads to data abstraction 
or data hiding. A real life example would be company having different sections 
like accounts section, sales, sections, etc. All the data for a single section is
binded or encapsulated together.\\

\texttt{Abstraction} : Abstraction means displaying only essential information 
and hiding the details. It is an important aspect of object oriented
programming. A real life example would be a person driving a car, since the person 
just knows about the pressing the accelerator or brake would do a job of
controlling speed but how the speed is exactly being controlled he never knows.
Abstraction can be achieved as follows : - 
\begin{enumerate}
    \item Abstraction by Header Files : Example in pow function in math.h library. 
          Also the data structures declared in c++ STL library.
    \item Abstraction by Class : Since class can decide which data member will be
          visible to the outside world since it can group the data members and 
          the member functions together.\\
\end{enumerate}

\texttt{Access Modifiers} : They are used to implement the most important thing
which is data hiding or abstraction. There are three types of access modifiers
in C++.
\begin{enumerate}
    \item \emph{Public} - The public members of a class can be accessed 
          from anywhere in the program using the direct member access operator (.) 
          with the object of that class.
    \item \emph{Private} - The class members declared as private can be accessed 
          only by the functions inside the class and are not allowed to be accessed 
          directly by any object.
    \item \emph{Protected} - Similar to private class but only difference is
          that they can access by any subclass or derived class function.\\
\end{enumerate}

\texttt{Inheritance} : It is a feature of Object Oriented Programming which 
has capability of a class to derive properties and characteristics from another 
class. There are five types of inheritance as follow - 
\begin{enumerate}
    \item \emph{Single Inheritance} - deriving a class from single base class.
    \item \emph{Multiple Inheritance} - deriving a class from multiple base class.
    \item \emph{Multilevel Inheritance} - deriving a class from another derived class.
    \item \emph{Hierarchical Inheritance} - more than one derived class exits
                from a single base class.
    \item \emph{Hybrid Inheritance} - Its a combination of the above mentioned
                inheritance.\\
\end{enumerate}

\texttt{Polymorphism} : The words means to have many froms. Real life example of
polymorphism, is that person at the same time can have different characteristic 
like he can be a student, son, grandchild, etc. There are two types of
polymorphism as follows - 
\begin{enumerate}
    \item \emph{Compile Time Polymorphism} : it is also called as early or
          static binding and futher can be classified as - 
          \begin{itemize}
              \item Function Overloading - same function different signatures.
              \item Operator Overloading - operators(+,-,*,/) can be used on objects.
          \end{itemize}
    \item \emph{Run Time Polymorphism} : it is late or dynamic binding and
          comes by implementation of virtual functions.
\end{enumerate}

\end{document}

