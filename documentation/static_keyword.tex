\documentclass[oops.tex]{subfiles}
\begin{document}
\section{Static keyword in C++}

We know that static variables are have \emph{property of preserving their value 
even after they are out of their scope}. A few properties of static keyword in
C language include - 
\begin{enumerate}
    \item Static variables remains in the memory while the program is running
          unlike the auto variables. The memory allocated to static variables
          is in the data segment.
    \item Static variables like global variables are initialized as 0 if not
          initialized explicitly.
    \item Static function access is restricted to the file where they are declared 
\end{enumerate}

Now lets discuss the static keyword used with class. We can declare static
data member as well as static member function in the class. Object of the class
can also be declared as static.

\begin{enumerate}
    \item {\bf Static variables in class} - They are initialized only once
          and they are allocated space in separate static storage. Thus we can
          say that static variables in class \texttt{are shared by the objects}.
          There \texttt{cannot be multiple copies of the static variables}
          \lstinputlisting[% Static variable in class 
          style=codeStyleC++, caption=Static variable in Class]
          {listings/staticKeyword/staticVariable.cpp}

          Notice that there is need of \texttt{explicitly defination} of static 
          variable outside class using the class name and the scope resolution
          operator.\\
          Now let us create objects and manipulate the static variable.
          \lstinputlisting[% Objects having static variables  
          style=codeStyleC++, caption=Objects with Static variable]
          {listings/staticKeyword/objectsStaticVariable.cpp}

    \item {\bf Static functions in a class} - Similar to static variables the
          static member functions are independent of the objects created. A
          static member function can be called even if no objects are created
          using the class name and the scope resolution operator.
          \lstinputlisting[% Static functions in class 
          style=codeStyleC++, caption=Static function in class]
          {listings/staticKeyword/staticFunction.cpp}

          \begin{itemize}
            \item   \texttt{Static member function can only access static data 
                    member or other static member functions of the class}. 
            \item   \texttt{Non static member functions in class 
                    can manipulate static data}.
          \end{itemize}
          \lstinputlisting[% Objects using static functions
          style=codeStyleC++, caption=Objects using static function] 
          {listings/staticKeyword/objectsStaticFunction.cpp}
    
    \item {\bf Class objects as static} - Objects are just like variables and
          thus can be declared as static and have the scope till the lifetime
          of the program. Lets create a class and make objects as static
          \lstinputlisting[% Class for instanciatiating objects
          style=codeStyleC++, caption=Simple class for instanciating static object] 
          {listings/staticKeyword/simpleClass.cpp}

          The following code creates static objects and given below.
          \lstinputlisting[% Static objects of a class
          style=codeStyleC++, caption=Static object of above class] 
          {listings/staticKeyword/staticObject.cpp}
          The output of the above code will be as given below - \\
          \texttt{Inside constructor}                           \\
          \texttt{End of main}                                  \\
          \texttt{Inside destructor}                            \\

\end{enumerate}

{\bf Why Static Variables inside Class ?}\\
In Langauges like \texttt{Java}, \emph{static} keyword is used for memory
management, example our code has many String objects created this static
varaible can keep count of many objects are allocated in memory at any
particular time in code. Also static keyword give lifetime scope for that
variable which makes it accessable whenever needed in code.


\end{document}

